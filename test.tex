\begin{document}
\maketitle
​
\section{本稿での議論の対象}
​
多次元正規分布$\mathcal{N}\left( \bm{\mu}, \Sigma \right)$に従う確率変数$\bm{X}_{all} = \{X_1, X_2, \cdots, X_n\}$を考える.
($\bm{X}_{all} \sim \mathcal{N}\left( \bm{\mu}, \Sigma \right)$と表す.)
​
一部の確率変数$\bm{O} \subseteq \bm{X}_{all}$の観測値$\bm{o}$が得られたとき,残りの確率変数$\bm{X} = \bm{X}_{all} \setminus \bm{O}$に対する事後確率分布$P\left( \bm{X} \mid \bm{O} = \bm{o} \right)$は多次元正規分布に従う.
​
\begin{equation}
  P\left( \bm{X} \mid \bm{O} = \bm{o} \right) \sim \mathcal{N}\left( \bm{\mu_{X \mid o}}, \Sigma_{\bm{X \mid o}} \right) \label{eq:ppd}
\end{equation}
\noindent また,この分布の期待値ベクトル$\bm{\mu_{X \mid o}}$,及び共分散行列$\Sigma_{\bm{X \mid o}}$はそれぞれ以下の通り求められる.
​
\begin{align}
  \bm{\mu_{X \mid o}} = & \bm{\mu_X} + \Sigma_{\bm{XO}}\Sigma_{\bm{OO}}^{-1}\left( \bm{o} - \bm{\mu_o} \right) \label{eq:ppd-mu}\\
  \Sigma_{\bm{X \mid o}} = & \Sigma_{\bm{XX}} - \Sigma_{\bm{XO}}\Sigma_{\bm{OO}}^{-1}\Sigma_{\bm{OX}} \label{eq:ppd-sigma}
\end{align}
\noindent ただし,各記号の添字はベクトルないし行列から添字に対応する次元を抽出したことを示す.
例えば,$\bm{\mu_{X}}$は$\bm{X}$に対応する次元の値を平均ベクトル$\bm{\mu}$から抽出したもの(つまり$\bm{X}$の期待値ベクトル)であり,$\Sigma_{\bm{XO}}$は分散共分散行列$\Sigma$の$\bm{X}$に対応する行から$\bm{O}$に対応する列を抽出したもの(つまり$\bm{X}$と$\bm{O}$の共分散行列)である.
以上の議論は既に様々な文献でなされており,Web上にも導出の記事が数多く存在する.
​
対して本稿での議論の対象は,一部の確率変数$\bm{O} \subseteq \bm{X}_{all}$が取る値が確率的に分かった場合,特に$\bm{O} \sim \mathcal{N}\left( \bm{\mu}^*, \Sigma^* \right)$である場合の確率変数$\bm{X}$の事後確率分布
​
\begin{equation}
  P\left( \bm{X} \mid \bm{O} \sim \mathcal{N}\left( \bm{\mu}^*, \Sigma^* \right) \right) \label{eq:ppd4d}
\end{equation}
\noindent である.以降ではこのような\textbf{確率分布に対する事後確率分布}は存在するのか,また存在するとしたらどのような分布になるのかについて検討する.
​
なお,妥協案としては分布を考慮せずに期待値のみに対する事後確率分布を扱う,すなわち
​
\begin{equation}
  P\left( \bm{X} \mid \bm{O} \sim \mathcal{N}\left( \bm{\mu}^*, \Sigma^* \right) \right) \simeq P\left( \bm{X} \mid \bm{O} = \bm{\mu}^* \right)
\end{equation}
\noindent として\Eq{\ref{eq:ppd}}--(\ref{eq:ppd-sigma})から計算する方法が考えられる.
しかし,この方法では分散の情報を落としてしまうため,$\bm{X}$に対する推定値の誤差保証が破綻してしまうと考えられるので望ましくない.
本稿では分布を正しく考慮する方針で検討を進める.
​
\section{「確率分布に対する事後確率分布」の考え方}
​
確率分布に対する事後確率分布を考える上でのスタートラインとして,確率変数$\bm{O}$が$\bm{o_1}$と$\bm{o_2}$のどちらかの値を等確率で取る,すなわち$P(\bm{O} = \bm{o_1}) = P(\bm{O} = \bm{o_2}) = 0.5$と分かった状況を考える.
このときの確率変数$\bm{X}$の事後確率分布は,それぞれの値を取る場合の事後確率分布を半々の割合で混ぜた以下の混合ガウス分布として考えられる.
​
\begin{align}
  P\left( \bm{X} \mid \bm{O} \right) = & P\left( \bm{O} = \bm{o}_1 \right)P\left( \bm{X} \mid \bm{O} = \bm{o}_1 \right) + P\left( \bm{O} = \bm{o}_2 \right)P\left( \bm{X} \mid \bm{O} = \bm{o}_2 \right) \nonumber \\
  = & 0.5 P\left( \bm{X} \mid \bm{O} = \bm{o}_1 \right) + 0.5P\left( \bm{X} \mid \bm{O} = \bm{o}_2 \right)
\end{align}
あああ。(これはalignが最長マッチになっていないかのテストです。「。」でエラーを出してもらう予定)
\noindent 一般化すれば,$\bm{O}$が取り得る各値$\bm{o}_i$について,$\bm{o}_i$に対する$\bm{X}$の事後確率分布$P\left( \bm{X} \mid \bm{O} = \bm{o}_i \right)$と$\bm{O}=\bm{o}_i$となる確率$P\left( \bm{O}=\bm{o}_i \right)$の積の総和を取れば良い.
​
よって,本題の$\bm{O} \sim \mathcal{N}\left( \bm{\mu}^*, \Sigma^* \right)$である場合は次式の通り求められると考えられる.
​
\begin{align}
  P\left( \bm{X} \mid \bm{O} \sim \mathcal{N}\left( \bm{\mu}^*, \Sigma^* \right) \right) = & \int_{-\infty}^{\infty} P\left( \bm{O}=\bm{o} \right)P\left( \bm{X} \mid \bm{O} = \bm{o} \right) d\bm{o} \nonumber \\
  = & \int_{-\infty}^{\infty} \mathcal{N}\left( \bm{o} \mid \bm{\mu}^*, \Sigma^* \right) \mathcal{N}\left( \bm{\mu_{X \mid o}}, \Sigma_{\bm{X \mid o}} \right) d\bm{o}
\end{align}
​
