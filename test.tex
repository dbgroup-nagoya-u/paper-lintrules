私は人間である.
%\chapter{クラスファイルの使い方}
%卒業研究報告・修士論文・博士論文用の{\LaTeX}クラスファイル``thesis.cls''の使い方を説明します.

%\section{クラスオプション}
%\today 現在以下の3種類のオプションがあります.
%\begin{itemize}
%  \item bachelor/master/doctor:タイトルなどの形式をそれぞれ卒業研究報告/修士論文/博士論文に整形して出力します.
%  \item english:英語で記述する際に指定します.
%  \item nofigure, notable:論文が図ないし表を含まない際に指定します(図目次/表目次が出力されなくなります).
%\end{itemize}

%\section{タイトル,著者などの指定}
%論文のタイトルなどは,``\textbackslash begin\{document\}''の前で以下のコマンドを用いて指定します.
%コマンドの説明の最後に示す(b, m, d)の記号は,その記号がいずれのフォーマットで使用されるかを示しています.
%つまり,(b, m)となっているコマンドは卒業研究報告及び修士論文のみで使用し,博士論文を書く際にはそのコマンドを省略できます.
%なお,英語で書く場合は``\textbackslash jp...''の代わりに``\textbackslash en...''コマンドの方を使用してください.
%\emph{ただし,修士論文では``\textbackslash jp...''及び``\textbackslash en...''両方の指定が必要です.}
%\begin{itemize}
%  \item \textbackslash setYear, \textbackslash setMonth:論文の提出年月(b, m, d).
%  \item \textbackslash studentNumber:学生番号(b, m)
%  \item \textbackslash jpTitle:論文の題目.タイトルページでは2行,概要ページでは1行で題目を表示したい場合は,任意引数``[]''で1行の題目を,通常の引数``''\{\}''で2行の(\textbackslash\textbackslash ありの)題目を入力してください.(b, m, d)
%  \item \textbackslash jpAuthor:著者(b, m, d)
%  \item \textbackslash jpAffiliationMain:所属の前半.学部学科名や研究科名など(b, m)
%  \item \textbackslash jpAffiliationSub:所属の後半.コース名や専攻名など(b, m)
%  \item \textbackslash jpAbstract:論文の概要(b, m, d)
%  \item \textbackslash jpKeyword:論文のキーワード(b)
%\end{itemize}