% このファイルをコピーして、新しい.texファイルを作成してください。
% その後プルリクエストに対応する文章を追加し、対象となるエラーが出る/出ないかどうかを確認してください。
% 新規ファイルはマージ前に削除してください。

% textlint
あいうえお。
かきくけこ.
zipファイルではありません
それっぽい。
ウェッブ

% draftcheck
$n log n$だ。
aaa.\cite{bbb:123}

% test content
関連発表論文を書く際は``\textbackslash publications''コマンドを記述した後,``\textbackslash section*\{...\}''及び``\textbackslash begin\{enumerate\}\ ... \textbackslash end\{enumerate\}''で記述してください.

\item \textbackslash jpKeyword:論文のキーワード(b)

\item bachelor/master/doctor:タイトルなどの形式をそれぞれ卒業研究報告/修士論文/博士論文に整形して出力します.

\today 現在以下の3種類のオプションがあります.

% このファイルをコピーして、新しい.texファイルを作成してください。
% その後プルリクエストに対応する文章を追加し、対象となるエラーが出る/出ないかどうかを確認してください。
% 新規ファイルはマージ前に削除してください。

% textlint
あいうえお。
かきくけこ.
zipファイルではありません
それっぽい。
ウェッブ

% draftcheck
$n log n$だ。
aaa.\cite{bbb:123}

% test content
関連発表論文を書く際は``\textbackslash publications''コマンドを記述した後,``\textbackslash section*\{...\}''及び``\textbackslash begin\{enumerate\}\ ... \textbackslash end\{enumerate\}''で記述してください.

\item \textbackslash jpKeyword:論文のキーワード(b)

\item bachelor/master/doctor:タイトルなどの形式をそれぞれ卒業研究報告/修士論文/博士論文に整形して出力します.

\today 現在以下の3種類のオプションがあります.
